\documentclass{article}  
\usepackage{array}
\input{../library/preamble}  
\input{../library/style}  
\begin{document}

In order for the a robot to accomplish the different task it will need to decode all the signals that are sent to it. These signals then need to be sent to the right parts (i.e. for the Zebro the signals that are meant for the arm need to be sent to the arm). For this task a microcontroller is perfect.

\subsection{Important microcontroller specifications}

A microcontroller is a microprocessor that is used to control all different kinds of electronics. A few of the more important specifications are the clock speed, power characteristics, interrupts and operating temperature. For the clock speed, mostly given in \#\# MHz, and together with the CPI, cycles per instruction, the performance can be given for a microcontroller. This means that it has a certain execution time for a program. This is of course important since if the microcontroller is too slow the robot will get slowed down or signals send back to the user arrive too late in order to make an appropriate decision. \\
The second important specifications are the power charateristics, this includes the supply voltage and current and the power dissipation. Escpecially for a robot that has a limited power source, a battery, it is not advisable to use a microcontroller that drains the power source quickly. This specification might interfere with the first specification of performance of the mincrocontroller, if there is limited power it might be neccesary to choose a microcontroller that has lower performance. However this is a design option that needs to be made when the time comes.\\
The third important specification is the way th microcontroller can handle interrupts. Since the microcontroller is used for decoding signals it receives it inevitable that is a signal will be sent from a subsystem to the microcontoller to send back to the user. These signals might  be asynchronous and therefore the microcontroller needs to be interrupted and react to the input correctly.\\
One more specification to take into consideration is the operating temperature. This is straightforward, the minimum and maximum temperature the microcontroller can handle. It will be safe to operate the microcontroller within this range, in other words the microcontroller should correctly in this temperature range. When the temperature is exceeding this range the microcontroller is not guaranteed to operate the way it should. If the temperatures get to extreme it might even get permantly damaged. So for some apllications it is a wise choice to take this specification into consideration.\\







\end{document}