\documentclass{article}  
\usepackage{array}
\input{../library/preamble}  
\input{../library/style}  
\begin{document}

Radar stands for radio detection and ranging. It is widely used today in military, telecommunication, medical and many other applications. It utilizes electromagnetic waves to sense the environment. In our case it could be used to detect objects and thus avoid collisions.

\subsection{How detect object with radar}

To detect objects, electromagnetic waves have to be emitted first. This is done by radiating electromagnetic energy from an antenna to propagate in space. When these electromagnetic waves collide with an object, the waves will be scattered in many directions. But some of these reflected waves will return to the radar antenna and can be detected. When these reflected waves are picked up by an antenna, the signal will be processed and the location plus possible other information about the object can be obtained. Because the emitted electromagnetic waves are usually sent in all directions, it is possible to detect objects in a certain radius of the radar. So it is not needed to aim the antenna

To determine the range of an object, the time it takes for a radar signal to propagate to the object and back has to be measured. Because electromagnetic waves travel at a constant speed, the speed of light, the distance to the object can be calculated.An advantage of radar is that it can measure this distance with great accuracy, even at long range. Because we need to detect objects at a relatively small range, high accuracy becomes even more important.



\end{document}