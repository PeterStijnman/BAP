\documentclass{article}  
\usepackage{array}
\input{../library/preamble}  
\input{../library/style}  
\begin{document}

An onboard navigation system is necessary to control a free-roaming robot. This navigation system is connected to the user through an user interface. From here the user is able to control the robot when the user wants to, otherwise the robot should be able to autonomously do tasks on its own.\\
To be able to accomplish this there are certain subsystems that need to be implemented on the robot. These subsytems are seen in the function tree in figure \ref{intro1}.

\begin{figure}[H]
	\centering
	\includegraphics[scale=0.3]{figures/functiontree}
	\caption{Function tree.}
	\label{intro1}
\end{figure}

These subsystems are explained throughout this literature report. From this report design choices will be made and these will be implemented on the platform called Zebro. First of the report will start with an explanation of the different methods of sensing the environment. After that the positioning system will be discussed and then the telecommunication link will be explained. In chapter six two different camera types will be analysed and compared. The last part that will contain information on microcontroller specifications.





\end{document}