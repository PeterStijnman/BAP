\documentclass{article}  
\usepackage{array}
\input{../library/preamble}  
\input{../library/style}  
\begin{document}

An onboard navigation system is necessary to control a free-roaming robot. This navigation system is connected to the user through an user interface. From here the user is able to control the robot when the user wants to, otherwise the robot should be able to autonomously walk between set points.\\
To be able to accomplish this there are certain subsystems that need to be implemented on the robot. These subsytems are seen in the function tree in blue in figure \ref{intro1}. The orange blocks are possible solutions to implement these subsystems.

\begin{figure}[H]
	\centering
	\includegraphics[scale=0.27]{figures/functiontree}
	\caption{Function tree. The blue blocks indicate the different subsystems and the orange indicate the different possible solutions. }
	\label{intro1}
\end{figure}

These subsystems are explained throughout this literature report. From this report design choices will be made and these will be implemented on the platform called Zebro. First of the report will start the subsystems needed on the zebro platform. After that the telecommunication system will be discussed and after that the user side will be explained.





\end{document}